\iffalse %% true for article, false for presentation
\documentclass[12pt]{article}
\usepackage{beamerarticle}
\usepackage[lmargin=1.3in,rmargin=1.3in,tmargin=0.5in,bmargin=0.5in,marginparwidth=1.5in]{geometry}
\else
\documentclass[ignorenonframetext]{beamer}
\fi
\usepackage[export]{adjustbox}
\usepackage{mathptmx}
\usepackage{natbib}
\usepackage{apalike}
\usepackage{multicol}
\usepackage{eso-pic}
\usepackage[T1]{fontenc}
\usepackage{tikz}
\usepackage[english]{babel}
\usepackage[latin1]{inputenc}
\usepackage{helvet}
\usepackage{graphicx}
\usepackage{xcolor}
\usepackage{amsmath}
\usepackage{enumitem}

\makeatletter
\gdef\SetFigFont#1#2#3#4#5{%
  \reset@font\fontsize{#1}{#2pt}%
  \fontfamily{#3}\fontseries{#4}\fontshape{#5}%
  \selectfont}%
\makeatother

\setlist[itemize]{itemsep=20pt,label=$\bullet$}

\usepackage{tikz}

\defbeamertemplate<article>{frame begin}{lined}{\par\noindent\rule{\textwidth}{1pt}\par}
\defbeamertemplate<article>{frame end}{lined}{\par\noindent\rule{\textwidth}{1pt}\par}

\newcounter{framebox}
\defbeamertemplate<article>{frame begin}{tikzed}{%
  \par\noindent\stepcounter{framebox}%
  \tikz[remember picture,overlay] %
  \path (-1ex,0) coordinate (frame top \the\value{framebox});}
\defbeamertemplate<article>{frame end}{tikzed}{%
  \hspace*{\fill}\tikz[remember picture,overlay] %
  \draw (frame top \the\value{framebox}) rectangle (1ex,0);\par}


\mode<article>
{
  \setlength{\parskip}{10pt}
  \setlength{\parindent}{0pt}

  \setbeamertemplate{frame begin}[lined]
  \setbeamertemplate{frame end}[lined]
%  \addtobeamertemplate{frame begin}{}{\setbox0=\bgroup}
%  \addtobeamertemplate{frame end}{\egroup}{}
  \usepackage{fancyhdr}
  \renewcommand{\headrulewidth}{0pt}
  \fancypagestyle{plain}{%
    \fancyhead[L]{}
    \fancyhead[C]{}
    \fancyhead[R]{}
    \fancyfoot[C]{}
    \fancyfoot[R]{\raisebox{6in}[0pt][0pt]{\hbox to 0pt{\hspace{0.3in}\framebox{\LARGE\thepage}}}}}
  \pagestyle{plain}
}

\newcommand{\cp}[1]{\textcolor{lightgray}{#1}}
\setcitestyle{authoryear,open={(},close={)}}

\mode<presentation>
{
%  \hypersetup{colorlinks=true,citecolor=green}

  \usecolortheme{default}
  \useinnertheme[shadow]{rounded}
  \useoutertheme{tsinfolines}

  \usesubitemizeitemtemplate{%
    \tiny\raise1.5pt\hbox{\color{beamerstructure}$\blacktriangleright$}%
  }
  \usesubsubitemizeitemtemplate{%
    \tiny\raise1.5pt\hbox{\color{beamerstructure}$\bigstar$}%
  }

  \setbeamersize{text margin left=1em,text margin right=1em}

% Turns off headline.
%  \setbeamertemplate{headline}[default]

  %% started as:  \usetheme[secheader]{Boadilla}
%  \usecolortheme{albatross}
  % or ...

  \setbeamercovered{invisible}
  % or whatever (possibly just delete it)

%  \addtobeamertemplate{frame title}{\begin{center}}{\end{center}}
%  Doesn't seem to do what I'd want.

}

%\beamertemplatenavigationsymbolsempty


\title{VR Graphics Programming for, well, you know}
\subtitle{Some (relatively) easy steps to virtual reality, Part 1}
\author{Tom Sgouros}
\institute{Center for Computation
    and Visualization\\Brown University\\thomas\_sgouros@brown.edu}
\date{Spring 2018}


% If you wish to uncover everything in a step-wise fashion, uncomment
% the following command:

%\beamerdefaultoverlayspecification{<+->}

\begin{document}

\maketitle

\begin{frame}
\titlepage
\end{frame}

\section{Basic 3D graphics}

\begin{frame}{Basic 3D graphics}
  \begin{description}
  \item[Coordinates] Where things are in space.
  \item[Transformation matrices] How one space relates to another.
  \item[Objects] Shape, color, location.
  \item[Scene graph] How one object relates to another.
  \item[Illumination] Lights and lighting.
  \item[Textures] Surface treatments.
  \end{description}
\end{frame}

\subsection{Coordinates}

\begin{frame}{Coordinates}
\begin{itemize}
\item Four dimensions, ``Homogeneous coordinates'' $[{\color{blue}x}, {\color{blue}y}, {\color{blue}z}, {\color{red}w}]$
\item Vertex locations, normal directions.

\item Also color (also 4D  $[{\color{red}r}, {\color{green}g},
  {\color{blue}b}, {\color{gray}a}]$), texture $[u.v]$, other?

\end{itemize}
\end{frame}


\subsection{Transformation matrices}

\begin{frame}{Spaces and transformation matrices}
\begin{itemize}
\item Move from one space to another.

\item Model space -- where an object is
  defined.\\\hspace{10em}$\Updownarrow$ {\small\color{red} model matrix}
\vspace{-1.8em}
\item World space -- where an object is placed.\\\hspace{10em}$\Updownarrow$ {\small\color{red} view matrix}
\vspace{-1.8em}
\item Camera space -- where an object appears to be placed (POV).\\\hspace{10em}$\Updownarrow$ {\small\color{red} projection matrix}
\vspace{-1.8em}
\item Screen -- where it appears on the screen
\end{itemize}
\end{frame}

\begin{frame}{Transformation matrices}
\begin{gather*}
\left[\begin{array}{cccc}
        \color{cyan}a & \color{blue}b & \color{blue}c & \color{red}j \\
        \color{blue}d & \color{cyan}e & \color{blue}f & \color{red}k \\
        \color{blue}g & \color{blue}h & \color{cyan}i & \color{red}l \\
        0 & 0 & 0 & 1
      \end{array}\right]\times\left[\begin{array}{c}x\\ y \\z \\w\end{array}\right]
  \end{gather*}

\hfill
{\color{red}Translation} \hfill
{\color{blue}Rotation} \hfill
{\color{green}Scale} \hfill~


\begin{align*}
\left[\begin{array}{cccc}x & y & z & 1\\\end{array}\right]&
~~~\text{Point} \\
\left[\begin{array}{cccc}x & y & z & 0\\\end{array}\right]&
~~~\text{Direction}
\end{align*}

\end{frame}

\subsection{Objects}

\begin{frame}[fragile]{drawableObjData}
A way to link some data, a name string, and two IDs (that OpenGL calls names).
\begin{verbatim}
template <class T>
class drawableObjData {
 private:
  std::vector<T> _data;

 public:
 drawableObjData(): name("") {
    _data.reserve(50); ID = 0; bufferID = 0;
  };
 drawableObjData(const std::string inName,
                 const std::vector<T> inData) :
  name(inName), _data(inData) {}
\end{verbatim}
\end{frame}

\begin{frame}[fragile]{drawableObj}
The data you need to draw an object.

\begin{verbatim}
class drawableObj {
 protected:

  // Specifies whether this is a triangle, a triangle strip,
  // fan, quads, points, whatever.
  GLenum _drawType;
  GLsizei _count;

  drawableObjData<glm::vec4> _vertices;
  drawableObjData<glm::vec4> _colors;
  drawableObjData<glm::vec4> _normals;
  drawableObjData<glm::vec2> _uvs;
\end{verbatim}
\end{frame}

\subsection{Scene graph}

\begin{frame}{Scene graph}
A way to group a collection of objects to be drawn into small groups
that can be manipulated as a single object.  Creates a hierarchy of
model matrices with object models placed on the tree.

\vspace{10pt}

\begin{minipage}{\columnwidth}
  % Copied in from scene-graph.latex, but with the \unitlength
  % adjusted slightly and the SetFigFont definition removed.  Not sure
  % why that part wouldn't work in situ, but it wouldn't.
\setlength{\unitlength}{3447sp}%
\begin{center}
\begin{picture}(5137,2587)(211,-1793)
\put(1951,614){\makebox(0,0)[lb]{\smash{{\SetFigFont{14}{16.8}{\rmdefault}{\mddefault}{\updefault}{\color[rgb]{0,0,0}Scene}%
}}}}
\thicklines
{\color[rgb]{0,0,0}\put(601,-1111){\line(-4,-5){300}}
}%
{\color[rgb]{0,0,0}\multiput(676,-1111)(-1.66667,-8.33333){46}{\makebox(6.6667,10.0000){\small.}}
}%
{\color[rgb]{0,0,0}\multiput(751,-1111)(3.75000,-7.50000){41}{\makebox(6.6667,10.0000){\small.}}
}%
{\color[rgb]{0,0,0}\put(826,-1111){\line( 1,-1){375}}
}%
{\color[rgb]{0,0,0}\put(1576,-436){\line( 5,-3){750}}
}%
{\color[rgb]{0,0,0}\put(1051,-436){\line(-4,-5){300}}
}%
\put(976,-361){\makebox(0,0)[lb]{\smash{{\SetFigFont{14}{16.8}{\rmdefault}{\mddefault}{\updefault}{\color[rgb]{0,0,0}Tractor}%
}}}}
\put(301,-1036){\makebox(0,0)[lb]{\smash{{\SetFigFont{14}{16.8}{\rmdefault}{\mddefault}{\updefault}{\color[rgb]{0,0,0}Wheels}%
}}}}
\put(226,-1711){\makebox(0,0)[lb]{\smash{{\SetFigFont{14}{16.8}{\rmdefault}{\mddefault}{\updefault}{\color[rgb]{0,0,0}1}%
}}}}
\put(526,-1711){\makebox(0,0)[lb]{\smash{{\SetFigFont{14}{16.8}{\rmdefault}{\mddefault}{\updefault}{\color[rgb]{0,0,0}2}%
}}}}
\put(901,-1711){\makebox(0,0)[lb]{\smash{{\SetFigFont{14}{16.8}{\rmdefault}{\mddefault}{\updefault}{\color[rgb]{0,0,0}3}%
}}}}
\put(1201,-1711){\makebox(0,0)[lb]{\smash{{\SetFigFont{14}{16.8}{\rmdefault}{\mddefault}{\updefault}{\color[rgb]{0,0,0}4}%
}}}}
\put(1351,-1036){\makebox(0,0)[lb]{\smash{{\SetFigFont{14}{16.8}{\rmdefault}{\mddefault}{\updefault}{\color[rgb]{0,0,0}Engine}%
}}}}
\put(2326,-1036){\makebox(0,0)[lb]{\smash{{\SetFigFont{14}{16.8}{\rmdefault}{\mddefault}{\updefault}{\color[rgb]{0,0,0}Seat}%
}}}}
{\color[rgb]{0,0,0}\multiput(3451,-436)(-3.12500,-7.81250){49}{\makebox(6.6667,10.0000){\small.}}
}%
{\color[rgb]{0,0,0}\put(3826,-436){\line( 1,-1){375}}
}%
{\color[rgb]{0,0,0}\multiput(3226,-1111)(-3.12500,-7.81250){49}{\makebox(6.6667,10.0000){\small.}}
}%
{\color[rgb]{0,0,0}\put(3376,-1111){\line( 4,-5){300}}
}%
{\color[rgb]{0,0,0}\multiput(4426,-1111)(3.12500,-7.81250){49}{\makebox(6.6667,10.0000){\small.}}
}%
{\color[rgb]{0,0,0}\put(4576,-1111){\line( 2,-1){750}}
}%
\put(3376,-361){\makebox(0,0)[lb]{\smash{{\SetFigFont{14}{16.8}{\rmdefault}{\mddefault}{\updefault}{\color[rgb]{0,0,0}Farmer}%
}}}}
\put(3076,-1036){\makebox(0,0)[lb]{\smash{{\SetFigFont{14}{16.8}{\rmdefault}{\mddefault}{\updefault}{\color[rgb]{0,0,0}Head}%
}}}}
\put(2851,-1711){\makebox(0,0)[lb]{\smash{{\SetFigFont{14}{16.8}{\rmdefault}{\mddefault}{\updefault}{\color[rgb]{0,0,0}Eyes}%
}}}}
\put(3526,-1711){\makebox(0,0)[lb]{\smash{{\SetFigFont{14}{16.8}{\rmdefault}{\mddefault}{\updefault}{\color[rgb]{0,0,0}Mouth}%
}}}}
\put(4201,-1036){\makebox(0,0)[lb]{\smash{{\SetFigFont{14}{16.8}{\rmdefault}{\mddefault}{\updefault}{\color[rgb]{0,0,0}Body}%
}}}}
\put(4501,-1711){\makebox(0,0)[lb]{\smash{{\SetFigFont{14}{16.8}{\rmdefault}{\mddefault}{\updefault}{\color[rgb]{0,0,0}Arms}%
}}}}
\put(5326,-1711){\makebox(0,0)[lb]{\smash{{\SetFigFont{14}{16.8}{\rmdefault}{\mddefault}{\updefault}{\color[rgb]{0,0,0}Legs}%
}}}}
{\color[rgb]{0,0,0}\put(2026,539){\line(-1,-1){675}}
}%
{\color[rgb]{0,0,0}\put(2251,539){\line( 3,-2){1038.461}}
}%
{\color[rgb]{0,0,0}\multiput(1426,-436)(3.12500,-7.81250){49}{\makebox(6.6667,10.0000){\small.}}
}%
\end{picture}%
\end{center}
\end{minipage}


\end{frame}

\begin{frame}[fragile]{drawableMulti}
A way to manage a position and location, and the model matrix
system for a scene graph.
\begin{verbatim}
class drawableMulti {
 protected:

  drawableMulti* _parent;

  std::string _name;

  glm::vec3 _position;
  glm::vec3 _scale;
  glm::quat _orientation;

  glm::mat4 _modelMatrix;
\end{verbatim}
\end{frame}

\begin{frame}[fragile]{drawableCompound}

An object made up of pieces that you don't expect to move relative to
one another.

\begin{verbatim}
class drawableCompound : public drawableMulti {
 protected:
  typedef std::list<bsgPtr<drawableObj> > DrawableObjList;
  DrawableObjList _objects;

  bsgPtr<shaderMgr> _pShader;

  glm::mat4 _totalModelMatrix;

  glm::mat4 _normalMatrix;

  std::string _modelMatrixName;
  GLuint _modelMatrixID;
\end{verbatim}
\end{frame}

\begin{frame}{bsgMenagerie}
  \begin{center}
    A fairly random collection of drawableCompound objects.

\vspace{10pt}

    \begin{minipage}{0.8\columnwidth}
      \begin{multicols}{2}
        \begin{description}
        \item[drawableRectangle]
        \item[drawableSquare]
        \item[drawableCube]
        \item[drawableSphere]
        \item[drawableCone]
        \item[drawableCircle]
        \item[drawableCylinder]
        \item[drawableAxes]
        \item[drawableLine]
        \item[drawableSaggyLine]
        \item[drawablePoints]
        \item[drawableObjModel]
        \end{description}
      \end{multicols}
    \end{minipage}
  \end{center}
\end{frame}

\begin{frame}[fragile]{drawableCollection}

A collection of drawableCompound and drawableCollection objects.  This
is the scene graph.

\begin{verbatim}
class drawableCollection : public drawableMulti {

  typedef std::map<std::string, bsgPtr<drawableMulti> >
                                            CollectionMap;
  CollectionMap _collection;

 public:
  std::string addObject(const std::string name,
               const bsgPtr<drawableMulti> &pMultiObject);
\end{verbatim}
\end{frame}

\begin{frame}{Scene graph redux}

  A look at the scene graph again.  The drawableCollection objects are
  the branches on this tree and drawableCompound objects are the
  leaves.  The drawableMulti is the parent class to both, but you
  won't instantiate any of them directly.

\vspace{10pt}

\begin{minipage}{\columnwidth}
  % Copied in from scene-graph.latex, but with the \unitlength
  % adjusted slightly and the SetFigFont definition removed.  Not sure
  % why that part wouldn't work in situ, but it wouldn't.
\setlength{\unitlength}{3447sp}%
\begin{center}
\begin{picture}(5137,2587)(211,-1793)
\put(1951,614){\makebox(0,0)[lb]{\smash{{\SetFigFont{14}{16.8}{\rmdefault}{\mddefault}{\updefault}{\color[rgb]{0,0,0}Scene}%
}}}}
\thicklines
{\color[rgb]{0,0,0}\put(601,-1111){\line(-4,-5){300}}
}%
{\color[rgb]{0,0,0}\multiput(676,-1111)(-1.66667,-8.33333){46}{\makebox(6.6667,10.0000){\small.}}
}%
{\color[rgb]{0,0,0}\multiput(751,-1111)(3.75000,-7.50000){41}{\makebox(6.6667,10.0000){\small.}}
}%
{\color[rgb]{0,0,0}\put(826,-1111){\line( 1,-1){375}}
}%
{\color[rgb]{0,0,0}\put(1576,-436){\line( 5,-3){750}}
}%
{\color[rgb]{0,0,0}\put(1051,-436){\line(-4,-5){300}}
}%
\put(976,-361){\makebox(0,0)[lb]{\smash{{\SetFigFont{14}{16.8}{\rmdefault}{\mddefault}{\updefault}{\color[rgb]{0,0,0}Tractor}%
}}}}
\put(301,-1036){\makebox(0,0)[lb]{\smash{{\SetFigFont{14}{16.8}{\rmdefault}{\mddefault}{\updefault}{\color[rgb]{0,0,0}Wheels}%
}}}}
\put(226,-1711){\makebox(0,0)[lb]{\smash{{\SetFigFont{14}{16.8}{\rmdefault}{\mddefault}{\updefault}{\color[rgb]{0,0,0}1}%
}}}}
\put(526,-1711){\makebox(0,0)[lb]{\smash{{\SetFigFont{14}{16.8}{\rmdefault}{\mddefault}{\updefault}{\color[rgb]{0,0,0}2}%
}}}}
\put(901,-1711){\makebox(0,0)[lb]{\smash{{\SetFigFont{14}{16.8}{\rmdefault}{\mddefault}{\updefault}{\color[rgb]{0,0,0}3}%
}}}}
\put(1201,-1711){\makebox(0,0)[lb]{\smash{{\SetFigFont{14}{16.8}{\rmdefault}{\mddefault}{\updefault}{\color[rgb]{0,0,0}4}%
}}}}
\put(1351,-1036){\makebox(0,0)[lb]{\smash{{\SetFigFont{14}{16.8}{\rmdefault}{\mddefault}{\updefault}{\color[rgb]{0,0,0}Engine}%
}}}}
\put(2326,-1036){\makebox(0,0)[lb]{\smash{{\SetFigFont{14}{16.8}{\rmdefault}{\mddefault}{\updefault}{\color[rgb]{0,0,0}Seat}%
}}}}
{\color[rgb]{0,0,0}\multiput(3451,-436)(-3.12500,-7.81250){49}{\makebox(6.6667,10.0000){\small.}}
}%
{\color[rgb]{0,0,0}\put(3826,-436){\line( 1,-1){375}}
}%
{\color[rgb]{0,0,0}\multiput(3226,-1111)(-3.12500,-7.81250){49}{\makebox(6.6667,10.0000){\small.}}
}%
{\color[rgb]{0,0,0}\put(3376,-1111){\line( 4,-5){300}}
}%
{\color[rgb]{0,0,0}\multiput(4426,-1111)(3.12500,-7.81250){49}{\makebox(6.6667,10.0000){\small.}}
}%
{\color[rgb]{0,0,0}\put(4576,-1111){\line( 2,-1){750}}
}%
\put(3376,-361){\makebox(0,0)[lb]{\smash{{\SetFigFont{14}{16.8}{\rmdefault}{\mddefault}{\updefault}{\color[rgb]{0,0,0}Farmer}%
}}}}
\put(3076,-1036){\makebox(0,0)[lb]{\smash{{\SetFigFont{14}{16.8}{\rmdefault}{\mddefault}{\updefault}{\color[rgb]{0,0,0}Head}%
}}}}
\put(2851,-1711){\makebox(0,0)[lb]{\smash{{\SetFigFont{14}{16.8}{\rmdefault}{\mddefault}{\updefault}{\color[rgb]{0,0,0}Eyes}%
}}}}
\put(3526,-1711){\makebox(0,0)[lb]{\smash{{\SetFigFont{14}{16.8}{\rmdefault}{\mddefault}{\updefault}{\color[rgb]{0,0,0}Mouth}%
}}}}
\put(4201,-1036){\makebox(0,0)[lb]{\smash{{\SetFigFont{14}{16.8}{\rmdefault}{\mddefault}{\updefault}{\color[rgb]{0,0,0}Body}%
}}}}
\put(4501,-1711){\makebox(0,0)[lb]{\smash{{\SetFigFont{14}{16.8}{\rmdefault}{\mddefault}{\updefault}{\color[rgb]{0,0,0}Arms}%
}}}}
\put(5326,-1711){\makebox(0,0)[lb]{\smash{{\SetFigFont{14}{16.8}{\rmdefault}{\mddefault}{\updefault}{\color[rgb]{0,0,0}Legs}%
}}}}
{\color[rgb]{0,0,0}\put(2026,539){\line(-1,-1){675}}
}%
{\color[rgb]{0,0,0}\put(2251,539){\line( 3,-2){1038.461}}
}%
{\color[rgb]{0,0,0}\multiput(1426,-436)(3.12500,-7.81250){49}{\makebox(6.6667,10.0000){\small.}}
}%
\end{picture}%
\end{center}
\end{minipage}


\end{frame}



\subsection{Illumination}

\begin{frame}[fragile]{lightMgr}

A list of lights and their colors.


\begin{verbatim}
class lightList {
 private:

  /// The positions of the lights in the list.
  drawableObjData<glm::vec4> _lightPositions;
  /// The colors of the lights in the list.
  drawableObjData<glm::vec4> _lightColors;
\end{verbatim}
\end{frame}

\subsection{Textures}

\begin{frame}[fragile]{textureMgr}

An easier way to load textures and make them available to a shaderhandle tex

\begin{verbatim}

class textureMgr {
 private:
  GLfloat _width, _height;

  GLuint _textureAttribID;
  std::string _textureAttribName;
  GLuint _textureBufferID;

  GLuint _loadPNG(const std::string imagePath);
  GLuint _loadCheckerBoard (const int size, int numFields);
  GLuint _loadTTF(const std::string ttfPath);
\end{verbatim}


\end{frame}

\end{document}
