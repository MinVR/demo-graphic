\iffalse %% true for article, false for presentation
\documentclass[12pt]{article}
\usepackage{beamerarticle}
\usepackage[lmargin=1.3in,rmargin=1.3in,tmargin=0.5in,bmargin=0.5in,marginparwidth=1.5in]{geometry}
\else
\documentclass[ignorenonframetext]{beamer}
\fi
\usepackage[export]{adjustbox}
\usepackage{mathptmx}
\usepackage{natbib}
\usepackage{apalike}
\usepackage{multicol}
\usepackage{fancyvrb}
\usepackage{eso-pic}
\usepackage[T1]{fontenc}
\usepackage{tikz}
\usepackage[english]{babel}
\usepackage[latin1]{inputenc}
\usepackage{helvet}
\usepackage{graphicx}
\usepackage{xcolor}
\usepackage{amsmath}
\usepackage{enumitem}

\makeatletter
\gdef\SetFigFont#1#2#3#4#5{%
  \reset@font\fontsize{#1}{#2pt}%
  \fontfamily{#3}\fontseries{#4}\fontshape{#5}%
  \selectfont}%
\makeatother

\setlist[itemize]{itemsep=20pt,label=$\bullet$}

\usepackage{tikz}

\defbeamertemplate<article>{frame begin}{lined}{\par\noindent\rule{\textwidth}{1pt}\par}
\defbeamertemplate<article>{frame end}{lined}{\par\noindent\rule{\textwidth}{1pt}\par}

\newcounter{framebox}
\defbeamertemplate<article>{frame begin}{tikzed}{%
  \par\noindent\stepcounter{framebox}%
  \tikz[remember picture,overlay] %
  \path (-1ex,0) coordinate (frame top \the\value{framebox});}
\defbeamertemplate<article>{frame end}{tikzed}{%
  \hspace*{\fill}\tikz[remember picture,overlay] %
  \draw (frame top \the\value{framebox}) rectangle (1ex,0);\par}


\mode<article>
{
  \setlength{\parskip}{10pt}
  \setlength{\parindent}{0pt}

  \setbeamertemplate{frame begin}[lined]
  \setbeamertemplate{frame end}[lined]
%  \addtobeamertemplate{frame begin}{}{\setbox0=\bgroup}
%  \addtobeamertemplate{frame end}{\egroup}{}
  \usepackage{fancyhdr}
  \renewcommand{\headrulewidth}{0pt}
  \fancypagestyle{plain}{%
    \fancyhead[L]{}
    \fancyhead[C]{}
    \fancyhead[R]{}
    \fancyfoot[C]{}
    \fancyfoot[R]{\raisebox{6in}[0pt][0pt]{\hbox to 0pt{\hspace{0.3in}\framebox{\LARGE\thepage}}}}}
  \pagestyle{plain}
}

\newcommand{\cp}[1]{\textcolor{lightgray}{#1}}
\setcitestyle{authoryear,open={(},close={)}}

\mode<presentation>
{
%  \hypersetup{colorlinks=true,citecolor=green}

  \usecolortheme{default}
  \useinnertheme[shadow]{rounded}
  \useoutertheme{tsinfolines}

  \usesubitemizeitemtemplate{%
    \tiny\raise1.5pt\hbox{\color{beamerstructure}$\blacktriangleright$}%
  }
  \usesubsubitemizeitemtemplate{%
    \tiny\raise1.5pt\hbox{\color{beamerstructure}$\bigstar$}%
  }

  \setbeamersize{text margin left=1em,text margin right=1em}

% Turns off headline.
%  \setbeamertemplate{headline}[default]

  %% started as:  \usetheme[secheader]{Boadilla}
%  \usecolortheme{albatross}
  % or ...

  \setbeamercovered{invisible}
  % or whatever (possibly just delete it)

%  \addtobeamertemplate{frame title}{\begin{center}}{\end{center}}
%  Doesn't seem to do what I'd want.

}

%\beamertemplatenavigationsymbolsempty


\title{VR Graphics Programming for, well, you know}
\subtitle{Some (relatively) easy steps to virtual reality, Part 4}
\author{Tom Sgouros}
\institute{Center for Computation
    and Visualization\\Brown University\\thomas\_sgouros@brown.edu}
\date{Spring 2018}


% If you wish to uncover everything in a step-wise fashion, uncomment
% the following command:

%\beamerdefaultoverlayspecification{<+->}

\begin{document}

\maketitle

\begin{frame}
\titlepage
\end{frame}


\section{OSCAR at CCV}

\subsection{General layout}

Lots of networked nodes, ssh between them, with ssh keys.

\begin{frame}{OSCAR cluster}

\begin{center}
\begin{minipage}{0.9\columnwidth}
\begin{itemize}

\item Lots of CPUs in lots of nodes.

\item Giant shared storage array: /gpfs

\item Heterogeneous array.  Some have graphics cards, some don't, old,
  new.

\item Identical OS image.

\item Separate authorization / authentication from rest of campus.

\item Mostly batch jobs through SLURM.

\end{itemize}
\end{minipage}
\end{center}
\end{frame}





\subsection{Modules}

why it exists

what it does

which ones you need.

.modules file

\begin{frame}[fragile]{Modules on OSCAR}

\begin{center}
\begin{minipage}{0.9\columnwidth}
\begin{itemize}

\item Support heterogeneous software.

\item Can't load all the specialized software onto images.

\item Controls environment variables like PATH, CPATH, LIBPATH,
  MANPATH, etc.

\end{itemize}
\begin{verbatim}
$ module load minvr
\end{verbatim}

\end{minipage}
\end{center}
\end{frame}


\begin{frame}{OSCAR accounts}

\begin{center}
\begin{minipage}{0.9\columnwidth}
\begin{itemize}

\item Quotas, myquota

\item data and scratch

\item ssh configuration

\item .modules file

\end{itemize}
\end{minipage}
\end{center}
\end{frame}





\begin{frame}

\begin{center}
\begin{minipage}{0.9\columnwidth}
\begin{itemize}

\item

\end{itemize}
\end{minipage}
\end{center}
\end{frame}





\subsection{YURT structure}


\begin{frame}{YURT structure}

\begin{center}

\begin{minipage}{0.9\columnwidth}
\begin{itemize}

\item Opti-trak tracker for head and hands

\item VRPN Button presses

\item 19 Linux machines, cave001, cave002, etc.

\item 69 video projectors

\item Scalable software applied to output.

\end{itemize}
\end{minipage}
\end{center}

\end{frame}

\begin{frame}[fragile]{MinVR Display Graph in practice}

\begin{Verbatim}[fontsize=\footnotesize]
<YURTGraph>
  <RootNode displaynodeType="VRGraphicsWindowNode" windowtoolkitType="VRFreeGLUTWindowToolkit" graphicstoolkitType="VROpenGLGraphicsToolkit">
    <LookAtNode displaynodeType="VRHeadTrackingNode">
      <StereoNode displaynodeType="VRStereoNode">
        <ScalableProjectionNode displaynodeType="VRScalableNode">
          <NearClip>0.01</NearClip>
          <FarClip>100.0</FarClip>
        </ScalableProjectionNode>
      </StereoNode>
    </LookAtNode>
  </RootNode>
</YURTGraph>
\end{Verbatim}
\end{frame}


\begin{frame}{Compiling for YURT}

\begin{center}
\begin{minipage}{0.9\columnwidth}
\begin{itemize}

\item Use login003 or login004

\item ssh3.ccv.brown.edu

\item VNC client (CCV page, Computing/Software, look left), use
  desktop3 or desktop4.


\item Don't compile on dev nodes or on login001 or login002.

\end{itemize}
\end{minipage}
\end{center}
\end{frame}




\begin{frame}{Controlling the YURT}

\begin{center}
\begin{minipage}{0.9\columnwidth}
\begin{itemize}

\item \texttt{pjcontrol} command.

\item Projectors numbered, list on paper near kiosk, or ops guide.
   http://github.com/tsgouros/yurt-ops

\item cave-utils module

\item \texttt{pjcontrol 0-68 on} also \texttt{off}, \texttt{error}

\item \texttt{pjcontrol 27 mono; sleep 10; pjcontrol 27 stereo}

\end{itemize}
\end{minipage}
\end{center}
\end{frame}

\begin{frame}{Oops}

\begin{center}
\begin{minipage}{0.9\columnwidth}
\begin{itemize}

\item cavesupport@ccv.brown.edu

\item support@ccv.brown.edu

\end{itemize}
\end{minipage}
\end{center}
\end{frame}








\end{document}
